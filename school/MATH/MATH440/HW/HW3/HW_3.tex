 
\documentclass {article}
\usepackage{amsfonts}
\usepackage{fancyhdr}
\usepackage{graphicx}


\setlength{\textwidth}{6.5in}
\setlength{\textheight}{8.5in}

\setlength{\oddsidemargin}{-.3in}
\setlength{\evensidemargin}{-.3in}

\setlength{\parindent}{0pt}
\setlength{\parskip}{12pt}

\usepackage{scrextend}



\def\dt        {{\Delta t}}
\def\dx        {{\Delta x}}

\def\eq        {\, = \,}
\def\gt        {\, > \,}
\def\implies   {{\, \Longrightarrow \,}}
\def\norm      {{| \! |}}

\begin{document}

\pagestyle{fancy}

\lhead{\bf HOMEWORK 3}


\rhead{\bf  MATH 440}
\noindent

\cfoot{\thepage}
 

%
%{\bf RUBRIC:}  
%
%
%\bigskip\bigskip
%\centerline{\begin{tabular}{|c||c|c|}\hline
%& & \\
%Questions & \quad Points \quad & \qquad Score \qquad\,\\ \hline\hline
%& & \\
% & & \\ \hline
%& & \\
%&  & \\ \hline
%& & \\
%&  & \\ \hline
%\hline
%& & \\
%Total &  & \\
%& & \\ \hline
%\end{tabular}}
%\newpage

%{\bf Problem 1.5.3(a-c):} Consider the polar coordinates
%\begin{eqnarray}
%x & = & r \cos \theta, \nonumber \\
%y & = & r \sin \theta. \nonumber
%\end{eqnarray}
%
%{\bf (a)} Since $r^2 = x^2+y^2$, show that $\frac{\partial r}{\partial x} = \cos \theta$, $\frac{\partial r}{\partial y} = \sin \theta$, $\frac{\partial \theta}{\partial y} = \frac{\cos \theta}{r}$, $\frac{\partial \theta}{\partial x} = -\frac{\sin \theta}{r}$.
%
%{\bf (b)} Argue that $\hat{r} = \cos \theta \hat{i}+ \sin \theta \hat{j}$ and $\hat{\theta} = -\sin \theta \hat{i} + \cos \theta \hat{j}$.
%
%{\bf (c)} Using the chain rule, show that $\vec{\nabla} = \hat{r} \frac{\partial}{\partial r} + \hat{\theta} \frac{1}{r} \frac{\partial}{\partial \theta}$.
%
%{\bf (d)} If $\vec{A} = A_r \hat{r} + A_{\theta} \hat{\theta}$,  show that $\vec{\nabla} \cdot \vec{A} = \frac{1}{r}\frac{\partial}{\partial r} \left( r A_r \right) + \frac{1}{r} \frac{\partial }{\partial \theta} \left( A_{\theta} \right)$.
%
%{\bf (e)} Show that $\nabla^2 u = \frac{1}{r} \frac{\partial}{\partial r} \left( r \frac{\partial }{\partial r} \right) + \frac{1}{r^2} \frac{\partial^2}{\partial \theta^2}$

{\bf Problem 1.5.10:}  Determine the equilibrium temperature distribution inside a circle $(r \le r_0)$ if the boundary is fixed at a temperature $T_0$.

{\bf Problem 1.5.11:}  Consider 
$$
\frac{\partial u}{\partial t} = \frac{k}{r} \frac{\partial }{\partial r} \left( r \frac{ \partial u}{\partial r} \right), \ \ \ a < r< b,
$$
subject to $u(r,0) = f(r)$, $\frac{\partial u}{\partial r} (a,t) = \beta$, and $\frac{\partial u}{\partial r}(b,t) = 1$.  For what value(s) of $\beta$ does an equilibrium temperature distribution exist?


{\bf Problem 2.2.1:}  Show that any linear combination of linear operators is a linear operator. 


{\bf Problem 2.2.2:}  Show that

{\bf (a)} $L(u) = \frac{\partial}{\partial x} \left[ K_0 (x) \frac{\partial u}{\partial x} \right]$ is a linear operator

{\bf (b)} and usually $L(u) = \frac{\partial }{\partial x} \left[ K_0(x,u) \frac{\partial u}{\partial x} \right]$ is not a linear operator.

{\bf Problem 2.2.3:}  Show that $\frac{\partial u}{\partial t} = k \frac{\partial^2u}{\partial x^2} + Q(u,x,t)$ is linear if $Q = \alpha(x,t)u + \beta(x,t)$ and, in addition, homogeneous if $\beta(x,t) = 0$.

{\bf Problem 2.2.4:}  In this exercise we derive superposition principles for non homogeneous problems.

{\bf (a)}  Consider $L(u) = f$.  If $u_p$ is a particular solution, $L\left( u_p\right) = f$, and if $u_1$ and $u_2$ are homogeneous solutions, $L\left(u_i\right) = 0$, show that $u = u_p+c_1u_1+c_2u_2$ is another particular solution.

{\bf (b)}  If $L(u) = f_1+f_2$, where $u_{pi}$ is a particular solution corresponding to $f_i$, what is a particular solution for $f_1+f_2$?

{\bf Problem 2.3.1 b,d:}  For the following partial differential equations, what ordinary differential equations are implied by the method of separation of variables?

{\bf (b)} $\frac{\partial u}{\partial t} = k \frac{\partial^2 u}{\partial x^2}-v_0 \frac{\partial u}{\partial x}$

{\bf (d)} $\frac{\partial u}{\partial t} = \frac{k}{r^2} \frac{\partial }{\partial r} \left( r^2 \frac{\partial u}{\partial r} \right)$

{\bf Problem 2.3.2 a,e:}  Consider the differential equation 
$$
\frac{d^2 \phi}{d x^2} + \lambda \phi = 0,
$$
where $\phi$ is a function of $x$ only.  Determine the eigenvalues $\lambda$ (and corresponding eigenfunctions) if $\phi$ satisfies the following boundary conditions.  Analyze three cases ($\lambda > 0$, $\lambda = 0$, and $\lambda<0$).  You may assume that the eigenvalues are real.

{\bf (a)} $\phi(0) = 0$ and $\phi (\pi) = 0$

%{\bf (b)} $\phi(0) = 0$ and $\phi ( 1) = 0$

{\bf (e)} $\frac{d \phi }{d x} (0) = 0$ and $\phi (L) = 0$

{\bf Problem 2.3.3 a,c,d:}  Consider the heat equation 
$$
\frac{\partial u}{\partial t} = k \frac{\partial^2 u}{\partial x^2},
$$
subject to the boundary conditions $u(0,t) = 0$ and $u(L,t) = 0$.  Solve the initial value problem if the temperature is initially 

{\bf (a)} $u(x,0) = 6 \sin \left( \frac{9 \pi x}{L}\right)$

{\bf (c)} $u(x,0) = 2 \cos \left( \frac{ 3 \pi x}{L} \right)$

{\bf (d)} $u(x,0) = \Big\{ 
\begin{array}{ll}
1 & 0 < x\le L/2 \\
2 & L/2 < x < L
\end{array}
$

{\bf Problem 2.3.4 a-b:}  Consider
$$
\frac{\partial u}{\partial t} = k \frac{\partial^2 u}{\partial x^2},
$$
subject to the boundary conditions $u(0,t) = 0$, $u(L,t) = 0$ and $u(x,0) = f(x)$.

{\bf (a)} What is the total heat energy in the rod as a function of time?

{\bf (b)} What is the flow of heat energy out to the rod at $x = 0$? at $x = L$?

\end{document}