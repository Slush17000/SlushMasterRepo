 
\documentclass {article}
\usepackage{amsfonts}
\usepackage{amsmath} %this is a package you need to make triple integrals and line integrals (the ones with the circle)
\usepackage{esint}%this is a package you need to make triple integrals and line integrals (the ones with the circle)
\usepackage{fancyhdr}
\usepackage{graphicx}


\setlength{\textwidth}{6.5in}
\setlength{\textheight}{8.5in}

\setlength{\oddsidemargin}{-.3in}
\setlength{\evensidemargin}{-.3in}

\setlength{\parindent}{0pt}
\setlength{\parskip}{12pt}

\usepackage{scrextend}


\def\dt        {{\Delta t}}
\def\dx        {{\Delta x}}

\def\eq        {\, = \,}
\def\gt        {\, > \,}
\def\implies   {{\, \Longrightarrow \,}}
\def\norm      {{| \! |}}

\begin{document}

\pagestyle{fancy}

\lhead{\bf HOMEWORK 5}


\rhead{\bf  MATH 440}
\noindent

\cfoot{\thepage}


{\bf Problem 3.2.2(d):}  For the following functions, sketch the Fourier series of $f(x)$ (on the interval $-L \le x \le L$) and determine the Fourier coefficients where 

$f(x) = \begin{cases} 0 & x<0 \\
x & x>0
\end{cases}$

{\bf Problem 3.2.3:}  Show that the Fourier series operation is linear.  That is, show that the Fourier series of $c_1 f(x) + c_2 g(x)$ is the sum of $c_1$ times the Fourier series of $f(x)$ and $c_2$ times the series of $g(x)$.

{\bf Problem 3.3.1(b):}  For the following functions, sketch $f(x)$, the Fourier series of $f(x)$, the Fourier sine series of $f(x)$, and the Fourier cosine series of $f(x) = 1+x$

{\bf Problem 3.3.2(b):}  For the following functions, sketch the Fourier sine series of $f(x)$ and determine its Fourier coefficients:
 
 $f(x) = \begin{cases} 1 & x < L/6 \\
3 & L/6 < x < L/2 \\
0 & x > L/2 \end{cases} $

{\bf Problem 3.3.3(b):}  For the following functions, sketch the Fourier sine series of $f(x)$.  Also, roughly sketch the sum of a finite number of nonzero terms (at least the first two)of the Fourier sine series:

{\bf (b)} $f(x) = \begin{cases} 1 & x < L/2 \\
 0 & x > L/2\end{cases}
 $ 

{\bf Problem 3.3.5(b):}  For the following functions, sketch the Fourier cosine series of $f(x)$ and determine its Fourier coefficients:

 $f(x) = \begin{cases} 1 & x < L/6 \\
3 & L/6< x< L/2 \\
0 & x > L/2 \end{cases}$

{\bf Problem 3.3.6(b):}  For the following function, sketch the Fourier cosine series of $f(x)$.  Also roughly sketch the sum of a finite number of nonzero terms (at least the first two) of the Fourier cosine series:

{\bf (b)} $f(x) = \begin{cases} 0 & x <L/2 \\
1& x>L/2\end{cases}$

{\bf Problem 3.3.7:}  Show that $e^x$ is the sum of an even and an odd function.

{\bf Problem 3.3.8(a-c):}  

{\bf (a)} Determine the formulas for the even extension of any function $f(x)$.  Compare to the formula for the even part of $f(x)$.

{\bf (b)} Do the same for the odd extension of $f(x)$ and the odd part of $f(x)$.

{\bf (c)} Calculate and sketch the four functions of parts (a) and (b) if 
$$
f(x) = \begin{cases} x & x>0 \\
x^2 & x<0.\end{cases}
$$

%{\bf Problem 3.3.14(a):} Consider a function $f(x)$ that is even around $x = L/2$.  Show that the odd coefficients ($n$ odd) of the Fourier cosine series of $f(x)$ on $0\le x \le L$ are zero.

{\bf Problem 3.4.4(b):}  Suppose that $f(x)$ and $df/dx$ are piecewise smooth. Prove that the Fourier cosine series of a continuous function $f(x)$ can be differentiated term by term.

{\bf Problem 3.4.8:}  Consider 
$$
\frac{\partial u}{\partial t} = k \frac{\partial^2u}{\partial x^2}
$$
subject to $\frac{\partial u}{\partial x }(0,t) = 0 = \frac{\partial u}{\partial x} (L,t)$ and $u(x,0) = f(x)$. Solve in the following way.  Look for the solution as a Fourier cosine series.  Assume that $u$ and $\partial u/\partial x$ are continuous and that $\partial^2 u /\partial x^2$ and $\partial u/\partial t$ are piecewise smooth.  Justify all differentiations of infinite series.

{\bf Problem 3.5.2(a-b):} 

{\bf (a)}  Using (3.3.11) and (3.3.12), obtain the Fourier cosine series of $x^2$.

{\bf (b)} From part (a), determine the Fourier sine series of $x^3$.


\end{document}